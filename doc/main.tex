\documentclass{report}
\usepackage[margin=1.0in]{geometry}
\usepackage[utf8]{inputenc}
\usepackage{textgreek}
\usepackage{gensymb}
\usepackage{graphicx}
\usepackage{listings}
\usepackage{amsmath}
\usepackage{amssymb}
\graphicspath{ {renders/} }



\begin{document}
\chapter{Reference for Regularized Models of Fourier Difference Coefficients}
\section{Description of Data}
These routines are meant to analyze time resolved crystallography data collected using the ratio method and X-ray free electron lasers. This particular implementation is focused on data collected by the Hekstra lab at the X-ray Pump Probe (XPP) endstation of the Linac Coherent Light Source (LCLS) during the fall of 2017. The general ideas evolved here should be applicable to future experiments with minor modifications to the source code. 

The data in question consist of integrated reflection intensities from X-ray diffraction images collected ongoniometer mounted samples as well as associated metadata. 12 still images were collected at each goniometer angle in the datasets. Of the 12 images, 8 were acquired 300 fs after the initiation of a 1 ps THz laser pulse. 

\section{Notation}
I am going to use some set notation to keep track of summation indices throughout this document. 


\subsection{Indices for Data Address}
We will adopt some conventions to describe indexing of integrated Bragg peak intensities, IPM data, and optimization variables.
Let the dataset consist of a number of images indexed by $i\in {1, 2, ... n}$.
Then the indexes shall be subdivided into sets as follows:
\break


\begin{center}
\begin{tabular}{r | p{0.5\textwidth}}
$R_r$ & The set of images in run (crystal) number k \\ \hline
$\phi_s$ & The set of images with an oscillation number s \\ \hline
$N$ & The set of on images \\ \hline
$F$ & The set of off images \\ \hline
$h\in H$ & Shorthand for the Miller index corresponding to a reflection in the unmerged lattice containing redundant, symmetry related indices ($H$). \\ \hline
$\{h\}$ & The equivalent Miller index to $h$ within the merged, reciprocal asymmetric unit.
\end{tabular} 
\end{center}


\subsection{Experimental Data and Metadata}
Our data consist mainly of integrated reflection intensities. 
Let the reflection with Miller index, $h$, from image $i$ be denoted $I_{h,i}$. 
Generically, I will use $n$ to refer to images during which the crystal has been excited by a laser and $f$ to denote those that have not. 
When it is not clear from other context whether an image is "pumped" or not, I will rely no this convention
$I_{h,n \in N \cap \phi_s \cap R_k}$ will refer to the set of intensities for the reflection with Miller index, $h$, acquired for crystal $k$ in orientation $s$ which have been pumped with a laser. 
$I_{h,f \in F \cap \phi_s \cap R_k}$ would be the corresponding set of dark images to the previous set.

The most salient metadata are those from the intensity position monitors (IPMs) which lie upstream of the sample position at XPP. 
These detectors are described in some detail in another document. 
Briefly, they consist of four diodes which measure backscattered photons from a silicon nitride membrane.
The four readings supply an estimate of beam intensity and can be used to estimate the position of the beam center. 
There are two IPMs for which we have metadata, called $IPM_2$ \& $IPM_3$. 
Let readings associated with image $i$ and $IPM_n$ be denoted $IPM_{n,i}$.
If the readings from the four diodes need be addressed individually, I will denote them as $T_{n, i}, B_{n, i}, L_{n, i}, R_{n, i}$ for the top, bottom, left and right diodes respectively. 
If the beam position estimates are required, I will use $X_{n,i}$, $Y_{n,i}$ for the beam X and Y positions estimated from $IPM_n$ for image $i$. 
I will use the expression $IPM_n,i$ to refer to the total intensity of the four diodes of detector $n$ for a given X-ray pulse associated with image $i$. 
That is to say, $IPM_{n,i} = T_{n,i}+B_{n,i}+L_{n,i}+R_{n,i}$.
I determined that that fluctuations of the beam center make $IPM_{n,i}$ a poor approximation of the photon flux incident on the crystal during acquisition of image $i$. 
Therefore, I will introduce the term $J_i$ to represent the true photon flux over the surface of the crystal. 
These terms are summarized in the table below.

\begin{center}
\begin{tabular}{r | p{0.5\textwidth}}
$T_{n,i}$ & The intensity of the top diode of $IPM_n$ associated with image $i$ \\ \hline
$B_{n,i}$ & The intensity of the bottom diode of $IPM_n$ associated with image $i$ \\ \hline
$L_{n,i}$ & The intensity of the left diode of $IPM_n$ associated with image $i$ \\ \hline
$R_{n,i}$ & The intensity of the right diode of $IPM_n$ associated with image $i$ \\ \hline
$IPM_{n,i}$ & The total intensity of all diodes of $IPM_n$ associated with image $i$ \\ \hline
$J_i$ & The actual photon flux through the crystal associated with image $i$ \\ 
\end{tabular} 
\end{center}

\subsection{Inference and Ratio Equations}
The goal of this inference package is to employ the equations for the ratio method \cite{coppens} to infer fourier difference map coefficients. 
According to the random diffuse excitation model, the observed intensity ratio between still images acquired at the same crystal orientation is approximately
\begin{equation}
\frac{I_{h,n}}{I_{h,f}} =  1 + 2P\frac{\Delta F_{\{h\}}} {F_{\{h\}}}
\end{equation}
where $P$ is the fraction of proteins that are excited by the laser. 
$F_{h}$ refers to the reference structure factor amplitude in the ground state.
This version of the ratio equation assumes equivalent photon flux between image $n$ and $f$. 
This is not the case for our experiment.
Therefore, I augment the equation to account for photon flux
\begin{equation}\label{eq:ratio}
\frac{I_{h,n}J_f}{I_{h,f}J_n} =  1 + 2P\frac{\Delta F_{\{h\}}} {F_{\{h\}}}.
\end{equation}
For convenience, I will assume $P=\frac{1}{2}$. 
This change only effects the width of the $\Delta F$ distribution. 
Because difference maps are thresholded relatively, the true value of $P$ has no impact on the map. 
By substituting and rearranging \eqref{eq:ratio}, 
\begin{equation}
I_{h,n} - I_{h,f} \frac{J_n}{J_f} ( 1 + \frac{\Delta F_{h}} {F_{h}}) = 0. 
\end{equation}
Which amounts to a set of equations that are satisfied by the experimental data. 
The intensities can be sumed to yield one constraint on $\Delta F$ per miller index, per orientation of a sample. 

\begin{equation}
\sum_{n\in \phi_s\cap N} I_{h,n} - \sum_{f\in \phi_s\cap F} I_{h,f} \frac{\sum_{n\in\phi_s\cap N} J_n}{\sum_{f\in\phi_s\cap F} J_f} ( 1 + \frac{\Delta F_{h}} {F_{h}}) = 0. 
\end{equation}
%TODO: Think of a way to fix the following shortcoming in this notation. 
%Ideally, the solution should not involve delta functions.
The above notation, does not explicitly express the notion that $J$ should not be counted for images whereinthe Miller index was not observed or was rejected as an outlier. 

Assuming a Gaussian error model, we can write a log likelihood for the pumped intensities as 
\begin{equation}\label{eq:ml}
\begin{aligned}
-\log P(I_{h,N}|I_{h,F},\Delta F, J) = \sum_{s,k,h}w_{s,k,h} \bigg[ \sum_{n\in R_k \cap \phi_s\cap N} I_{h,n} - \sum_{f\in R_k \cap \phi_s\cap F} I_{h,f} \frac{\sum_{n\in R_k \cap \phi_s\cap N} J_n}{\sum_{f\in R_k \cap \phi_s\cap F} J_f} ( 1 + \frac{\Delta F_{h}} {F_{h}}) \bigg]^2  \\
w_{s,k,h} = \bigg[\sum_{n\in R_k\cap \phi_s\cap N}\sigma^2_{I_{h,n}} + \
             \bigg(\frac {\sum_{n\in R_k \cap \phi_s\cap N} J_n}{\sum_{f\in R_k \cap \phi_s\cap F} J_f}\bigg)^2\
             \bigg(1 + \frac{\Delta F_h}{F_h}\bigg)^2 \
             \sum_{n\in R_k\cap \phi_s\cap F}\sigma^2_{I_{h,f}}\bigg]^{-2}
\end{aligned}
\end{equation}

In the reference implementation, \eqref{eq:ml} is maximized over $\Delta F$ and $J$. However, naively optimizing this equation yields, very badly overfit parameters. Therefore, two regularizers are introduced to combat this, and the proper optimization problem is
\begin{equation}
\underset{\Delta F,J}{argmin} \quad -\log P(I_{h,N}|I_{h,F}, \Delta F, J) + \rho ||J - IPM_2||_2^2 + \lambda ||\Delta F||_2^2
\end{equation}
which enforces a Guassian prior on photon flux and the fourier difference coefficients. 
The strengths of the regularizers, $\rho$ and $\lambda$ can be chosen by crossvalidation. 


\end{document}
